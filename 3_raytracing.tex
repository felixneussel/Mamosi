\section{Implementing Raytracing}
\begin{frame}{Raytracing Basics}
\begin{columns}
\column{0.6\textwidth}
    \begin{itemize}
   \item<1-> 3-D-coordinate system
   \item<2-> Define:
   \begin{itemize}
   	\item Geometrical objects: Spheres, Triangles, ...
   	\item Screen $=$ rectangle
    \item Camera position $x_{\text{camera}}$
       
   \end{itemize}
   \item<3-> Grid of points on screen $\Leftrightarrow$ pixels of image
   \item<4-> Pixel Color $=$ Vector of red, green, blue intensities
   \item<5-> \textbf{Goal:} Compute color of each pixel
   \end{itemize}
   \column{0.4\textwidth}<1->
   \begin{figure}
                \centering
                \includegraphics[width=0.7\linewidth]{Felix_1.png}
            \end{figure}
\end{columns}
\end{frame}

\begin{frame}{Visibility Problem}
    \begin{columns}
        \column{0.6\textwidth}
        \begin{itemize}
            \item<1-> For each grid point $o\in\R^3$, define 
            $$
            \text{Direction: }d := o - x_{\text{camera}}
            $$
            $$
            \text{Ray: } \{o + t d \ : \ t > 0\}
            $$
            \item<2-> For each object, compute intersection with ray
            \item<3-> No intersection $\Rightarrow$ Pixel color $=$ background color (e.g. black)
            \item<4-> $\exists$ intersection $\Rightarrow$ Pixel color $=$ color of closest object
        \end{itemize}
        \column{0.4\textwidth}<1->
        \begin{figure}
                \centering
                \includegraphics[width=0.6\linewidth]{Felix_2.png}
            \end{figure}
    \end{columns}
\end{frame}

\begin{frame}{Visibility - Results}
    \begin{figure}
                \centering
                \includegraphics[width=0.5\linewidth]{only_visibility.png}
            \end{figure}
\end{frame}

\begin{frame}{Shadows}
    \begin{columns}
        \column{0.6\textwidth}
        \begin{itemize}
        \item<1-> Light source position: $x_{\text{light}}\in\R^3$
        \item<2-> Intersection point: $x_{\text{inter}}$
        \item<3-> Define
        $$
        \text{Shadow direction: }d_{\text{shadow}} := x_{\text{light}} - x_{\text{inter}}
        $$
        $$
        \text{Shadow ray: }\{x_{\text{inter}} + t d_{\text{shadow}} \ : \ t > 0\}
        $$
        \item<4-> Shadow ray intersects other object $\Rightarrow$ $x_{\text{inter}}$ lies inside shadow
        \item<5-> $\Rightarrow$ Pixel color $=$ background color
        \end{itemize}
        \column{0.4\textwidth}<1->
        \begin{figure}
                \centering
                \includegraphics[width=0.55\linewidth]{Felix_3.png}
            \end{figure}
    \end{columns}
\end{frame}

\begin{frame}{Shadows - Results}
    \begin{figure}
                \centering
                \includegraphics[width=0.5\linewidth]{visibility_shadows.png}
            \end{figure}
\end{frame}

\begin{frame}{Shading}
    \begin{columns}
        \column{0.6\textwidth}
        \begin{itemize}
        \item<1-> Brightness depends on: 
        \begin{itemize}
            \item Distance to light source
            \item Angle $\theta$ between shadow ray and surface normal $n$
            \item $I_s > 0$: Intensity of light source
        \end{itemize}
        \item<2-> Compute illumination factor 
        $$
        I_s \frac{\cos \theta}{\|x_{\text{inter}} - x_{\text{light}}\|^2}
        $$
        \item<3-> Shaded pixel color $=$ illumination factor $\cdot$ pixel color
        \end{itemize}
        \column{0.4\textwidth}<1->
        \begin{figure}
                \centering
                \includegraphics[width=0.7\linewidth]{Felix_4.png}
            \end{figure}
    \end{columns}
\end{frame}

\begin{frame}{Shading - Results}
    \begin{figure}
                \centering
                \includegraphics[width=0.5\linewidth]{raytracing_shading_new.png}
            \end{figure}
\end{frame}

%\subsection{Multiple light sources}

%    \begin{frame}{Multiple Light Sources - Theory}
%        \begin{columns}[T]
%            \column{0.5\textwidth}
%            \textbf{Physics}
%            \begin{figure}
%                \centering
%                \includegraphics[width=0.5\linewidth]{Multiple lights.png}
%            \end{figure}
%            \pause
%            \begin{equation*}
%                L_t = \frac{L_1}{\Delta x_1^2} \cos(90^{\circ} -\alpha) + \frac{L_2}{\Delta x_2^2} \cos(90^{\circ} -\beta)
%            \end{equation*}
%    
%            \column{0.5\textwidth}
%            \textbf{Model}
%            \pause
%            \vspace{0.5cm}
%            \begin{columns}[T]
%                \column{0.5\textwidth}
%                Datatype RGB: $
%                    \begin{bmatrix}
%                        R\\
%                        G\\
%                        B\\
%                    \end{bmatrix} $
    
%                \column{0.5\textwidth}
%                \begin{itemize}
%                    \item 0 for black
%                    \item 1 for white
%                    \item Brightness is linear
%                \end{itemize}
%            \end{columns}
%            \vspace{1cm}
%            \pause
%            \begin{equation*}
%                \text{Pixel} = \text{Color} \cdot L_t
%            \end{equation*}
%        \end{columns}
%    \end{frame}
    
%    \begin{frame}{Multiple Light Sources - Results}
%        \begin{figure}
%            \centering
%            \includegraphics[width=0.5\linewidth]{raytracing_multiple_lights.png}
%        \end{figure}
%    \end{frame}

%\subsection{Indirect diffusive illumination}

    \begin{frame}{Indirect Diffusive Illumination - Theory}
        \begin{columns}[T]
            \column{0.5\textwidth}
            \textbf{Physics}
            \begin{figure}
                \centering
                \includegraphics[width=0.55\linewidth]{Diffusive light.png}
            \end{figure}
            \pause
            \begin{equation*}
                L = \sum^\infty_{k=0}T^k L_e
            \end{equation*}
            
            \column{0.5\textwidth}
            \textbf{Model}

            \pause
            \begin{columns}[T]
                \column{0.5\textwidth}
                \begin{itemize}
                    \item Direct illumination
                \end{itemize}

                \column{0.5\textwidth}
                \begin{itemize}
                    \item Indirect illumination
                    \begin{itemize}
                        \pause
                        \item Diffusive
                        \item Specular
                    \end{itemize}
                \end{itemize}
            \end{columns}

            \pause
            \begin{itemize}
                \item Diffusion not captured with Ray Tracing
                \pause
                \item Modeled with constant term $L_0$
            \end{itemize}            

            \vspace{0.2cm}

            \begin{equation*}
                L = L_0 + \sum^\infty_{k=0}T_0^k L_e
            \end{equation*}

            \vspace{0.2cm}

            \pause
            \begin{equation*}
                \text{Pixel} = (L_t + L_0) \cdot \text{Color}
            \end{equation*}
        \end{columns}
    \end{frame}

    \begin{frame}{Indirect Diffusive Illumination - Results}
        \begin{figure}
            \centering
            \includegraphics[width=0.5\linewidth]{raytracing_ambient_light.png}
        \end{figure}  
    \end{frame}

%\subsection{Indirect specular illumination}

    \begin{frame}{Indirect Specular Illumination - Theory}
        \begin{columns}[T]
            \column{0.5\textwidth}
            \textbf{Physics}
            
            \begin{figure}
                \centering
                \includegraphics[width=0.5\linewidth]{Reflections.png}
            \end{figure}
            
            \pause
            \begin{itemize}
                \item Beam gets weaker over time
                \pause
                \item Reflected radiance is object specific
            \end{itemize}
            
            \column{0.5\textwidth}
            \textbf{Model}

            \pause
            \begin{figure}
                \centering
                \includegraphics[width=0.8\linewidth]{Reflections_model.png}
            \end{figure}

            \pause
            \begin{itemize}
                \item Maximum reflection depth
                \pause
                \item Object specific reflection coefficient
            \end{itemize}

            \pause
            \begin{equation*}
                \text{Pixel} = (L_t + L_0) \cdot \text{Color} + c_r \cdot L_r
            \end{equation*}
        \end{columns}
    \end{frame}

    \begin{frame}{Indirect Specular Illumination - Results}
        \begin{figure}
            \centering
            \includegraphics[width=0.5\linewidth]{raytracing_reflection.png}
        \end{figure}
    \end{frame}