\section{Mathematical Model}
\begin{frame}{Global Illumination Model}

\begin{columns}
\column{0.65\textwidth}
\begin{itemize}
\item<1-> physical model: radiometry 
\begin{itemize}
\item<1-> light = beam 
\end{itemize}
\item<2-> most important: Radiance $L(x, \omega$)
\begin{itemize}
    %\item radiant flux per unit area and per solid angle
    \item<2-> brightness of light beam
    \item<2-> if beam not interrupted $\Rightarrow$ radiance constant
\end{itemize}
\item<3-> global illumination $\rightarrow$ interaction between all objects in a scene 
\item<4-> takes into account:
\begin{itemize}
    \item<4-> direct light
    \item<4-> reflection
    \item<4-> refraction
\end{itemize}
\end{itemize}
\column<2->{0.35\textwidth}
\begin{figure}
\includegraphics[width=0.5\linewidth]{Reflections.png}
\caption{Illustration of radiance $L(x,\omega)$ (taken from \cite{Pharr.2023})}
\end{figure}
\end{columns}

\end{frame}

\begin{frame}{The Reflectance Equation}
\begin{columns}
\column{0.65\textwidth}
\begin{itemize}
    \item<1-> $x$ $=$ point in surface of object in scene
    \item<2-> reflected radiance in $x$ in the direction $\omega_r$: 
\onslide<3->
    \begin{equation}
L_r(x, \omega_r) = \int_{H^2} f_r(x, \omega_i \rightarrow \omega_r) L_i(x, \omega_i) \cos \theta_i \mathrm{d}\omega_i
\end{equation}
    \item<4-> Bidirectional Distribution Function $f_r(x, \omega_i \rightarrow \omega_r) $
    \item<4-> $\cos \theta_i$ from Lambert's Law
\end{itemize}
\column<2->{0.35\textwidth}
\begin{figure}
\includegraphics[scale = 0.48]{reflectance.png}
\end{figure}
\end{columns}
\end{frame}

\begin{frame}{Bidirectional Distribution Function (BRDF)}
\begin{equation}
\mathrm{d}L_r = f_r(x, \omega_i \rightarrow \omega_r) \mathrm{d}E
\end{equation}
\begin{itemize}
	\pause
    \item  $dE$ $\rightarrow$ incoming light intensity from direction $\omega_i$ 
    \pause
    \item $\mathrm{d}L_r$ $=$ outgoing radiance in direction of $\omega_r$
    \pause
\end{itemize}
\begin{center}
\begin{figure}
\includegraphics[scale = 0.4]{BRDF.png}
\caption{Different types of reflection (taken from \cite{Bungartz.2014})}
\end{figure}
\end{center}
\end{frame}

\begin{frame}{Lambert's Law}
\begin{center}
\begin{figure}
    \includegraphics[scale = 0.5]{lambert.png}
    \caption{Illustration of Lamberts's Law (taken from \cite{Pharr.2023})}
\end{figure}
\end{center}
\end{frame}

\begin{frame}{The Rendering Equation}
\begin{itemize}
\pause
\item energy conservation $\Rightarrow$
\begin{equation}
L_o(x, \omega_o) =  L_e(x, \omega_o) +  L_r(x, \omega_o)
\end{equation}
\pause
\item insert reflectance equation into (3) $\Rightarrow$
\begin{equation} L_o(x, \omega_o) =  L_e(x, \omega_e) + \int_{H^2} f_r(x, \omega_i \rightarrow \omega_r) L_i(x, \omega_i) \cos \theta_i \mathrm{d}\omega_i 
\end{equation}
\pause
\item light sources  $\rightarrow$ self emitting surfaces
\end{itemize}
\end{frame}

\begin{frame}{Solving the Rendering Equation}
\begin{itemize}
\item analytical solution 
\pause
\item approximate solution $\rightarrow$ only consider specific kind of light interaction
\pause
\item ray tracing $\rightarrow$ only accounts for ideally specular reflection between objects
\begin{itemize}
	\pause
	\item direct light $\rightarrow$ separate additional step
    \pause
    \item  $\Rightarrow$ integral 'vanishes'
    \pause
    \item maximal recursion depth $K$
    \pause
\end{itemize}
\end{itemize}
\end{frame}

